\chapter{Quasiclassical theory}
\section{Propagators}
In here: define notation, propagators, and matrix current.
Make it clear what approximations we have performed.
\section{Matrix current}
%Usadel diffusion equation:
%\begin{equation}
%  iD \nabla\cdot\mathbf{I} = U 
%\end{equation}
%Boundary conditions:
%\begin{equation}
%  \mathbf{I}_a = \mathbf{I}_b = [g_a, g_b]
%\end{equation}

\section{Novel notation}
Regarding the notation... How about simply saying fuck special notation for spin and nambu space? I mean, the reason we're having problems with overnotation is:
\begin{itemize}
  \item 2×2 matrices in spin space has underline-notation;
  \item 2×2 matrices in nambu space has hat-notation;
  \item 4×4 matrices in spin-nambu space has underline+hat notation;
  \item 8-element vectors in our rho-space has vec notation;
  \item The resulting 8×4×4 basis vector $\rho$ should get vec+underline+hat notation;
  \item We haven't chosen a notation for 8×8 matrices in rho-space yet...
\end{itemize}
And so on. It becomes chaos, and at this point, all the extra notation will rather serve to confuse the reader than help the reader grok what we're doing. Also,
the more special notation of this kind, the larger the risk of mistakes -- and a single mistake in all of these hats and underlines will cause a lot of confusion for the reader.

So how about this: we drop as much annotation as possible, and just use explicit definitions instead, like explicitly saying "we define the 8-element vector..." or "define the 8×8 matrix" etc instead of implying this from notation.

First of all, fuck non-quasiclassical quantities. All our equations are quasiclassical anyway, so we might as well not reserve e.g. $\check{G}$ for the non-quasiclassical propagator. 
So we define the Green's function etc as:
\begin{align}
\bm{G} &= \begin{bmatrix}\bm{G}^R & \bm{G}^K \\ 0 & \bm{G}^A \end{bmatrix}, &
\bm{J} &= \begin{bmatrix}\bm{J}^R & \bm{J}^K \\ 0 & \bm{J}^A \end{bmatrix}, &
\bm{U} &= \begin{bmatrix}\bm{U}^R & \bm{U}^K \\ 0 & \bm{U}^A \end{bmatrix}.
\end{align}
No underline, no check, no hat. Just bold letters, with a superscript that mentions R/K/A as usual.
Then we use lower-case letters for the spin-space decomposition:
\begin{align}
\bm{G}^R &= \begin{bmatrix}+\bm{g}^R & +\bm{f}^R \\ -\tilde{\bm{f}}^R & -\tilde{\bm{f}}^R \end{bmatrix}, &
\bm{J}^R &= \begin{bmatrix}+\bm{j}^R & +\bm{i}^R \\ -\tilde{\bm{i}}^R & -\tilde{\bm{j}}^R \end{bmatrix}, &
\bm{U}^R &= \begin{bmatrix}+\bm{u}^R & +\bm{v}^R \\ -\tilde{\bm{v}}^R & -\tilde{\bm{u}}^R \end{bmatrix}. &
\end{align}
When we define these, we explicitly say that all the lower-case quantities are 2×2 matrices in spin-space.
Then we explicitly define the rhos as:
\begin{align*}
  \bm\rho_0 &= \text{diag}(+\bm\sigma_0, +\bm\sigma_0^*), &
  \bm\rho_4 &= \text{diag}(+\bm\sigma_0, -\bm\sigma_0^*), \\
         &\;\;\vdots                             &
         &\;\;\vdots                             \\
  \bm\rho_3 &= \text{diag}(+\bm\sigma_3, +\bm\sigma_3^*), &
  \bm\rho_7 &= \text{diag}(+\bm\sigma_3, -\bm\sigma_3^*), \\
\end{align*}
where we mention that $\sigma_0$ is the identity and $\sigma_n$ the 2×2 Pauli matrices.
Then we say that we group these into an 8-element vector:
\begin{equation}
  \vec{\bm\rho} = [\bm\rho_0, \ldots, \bm\rho_7]
\end{equation}
At this point, we introduce a single new notation: things with a vector arrow have a structure in the 8-dimension spin-nambu vector space defined here.
We then define the new notation for the distribution function and currents in this new space:
\begin{align}
  \bm{H} &= \vec{H} \cdot \vec{\bm{\rho}}, &
  \bm{J} &= \vec{J} \cdot \vec{\bm{\rho}}.
\end{align}
So we dot two vector quantities, namely the 8-element vector and the 8×4×4-element basis vector, and get a 4×4-element matrix out.
No bold for the 8-element vectors, as they're in a different space.
I'm using a capital letter for the distribution function to follow the implicit ``lower-case letters are spin-space'' logic.
The inverse transformation is:
\begin{align}
  \vec{H} &= \text{Tr}[ \bm{H} \vec{\bm{\rho}} ] / 4, &
  \vec{J} &= \text{Tr}[ \bm{J} \vec{\bm{\rho}} ] / 4
\end{align}
So we "trace away" the matrix structure to map the distribution function onto our 8-element vector space.
Once we have two 8-element vector quantities, we define an 8×8 relationship between them using tensor notation:
\begin{align}
  \vec{J} = \tensor{M} \vec{H}
\end{align}
