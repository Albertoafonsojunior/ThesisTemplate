\chapter{Kinetic equation}
\section{Introduction}
After we perform a Pauli-decomposition, the part of the Usadel equation describing the nonequilibrium distribution function~$\BC{H}$ -- also known as the \emph{kinetic equation} -- can be written [Eq.~(12) in our arXiv manuscript]:
\begin{equation}
  \BC{U} = -\partial_z \BC{J}
\end{equation}
Meanwhile, the nonequilibrium matrix current becomes [Eq.~(21)]:
\begin{equation}
  \BC{J} = \BC{M}\partial_z\BC{H} + \BC{Q}\BC{H}
\end{equation}
Substituting the latter into the former we get [Eq.~(39)]:
\begin{equation}
  \BC{M} \partial_z^2 \BC{H} = -\big[ (\partial_z\BC{M}) + \BC{Q} \big] \partial_z \BC{H} - (\partial_z \BC{Q}) \BC{H} - \BC{U}
\end{equation}
What I will show here, is that we can in general write $\BC{U} = \BC{K} \BC{H}$, so that:
\begin{equation}
  \BC{M} \partial_z^2 \BC{H} = -\big[ (\partial_z\BC{M}) + \BC{Q} \big] \partial_z \BC{H} - \big[ (\partial_z \BC{Q}) + \BC{K} \big] \BC{H}
\end{equation}
If we then define the quantities:
\begin{align}
  \BC{C}_1 &= -\BC{M}^{-1} \big[ (\partial_z\BC{M}) + \BC{Q} \big] \\
  \BC{C}_2 &= -\BC{M}^{-1} \big[ \, (\partial_z\BC{Q}) \,+ \BC{K} \big]
\end{align}
Then the kinetic equation can be written simply as:
\begin{equation}
  \partial_z^2 \BC{H} = \BC{C}_1 \partial_z \BC{H} + \BC{C}_2 \BC{H}
\end{equation}
Here, the matrices $\BC{C}_1$ and $\BC{C}_2$ are determined solely by equilibrium properties, and thus can be considered constants when solving the nonequilibrium equation above.
This should make this form of the equation very efficient to solve, since it is an \emph{explicit and linear differential equation}.

\clearpage
\section{First-order terms}
First, I will look at terms which are first-order in the $8\times8$ propagator~$\UV{G}$.
Before Pauli-decomposition, the Usadel equation for such terms look like
\begin{equation}
  iD \partial_z(\UV{G} \partial_z \UV{G}) = \big[ \UV{\Sigma}, \UV{G} \big],
\end{equation}
where I first assume that $\UV{\Sigma}$ is independent of $\UV{G}$, making the right-hand side above first-order in $\UV{G}$.
Upon introducing $\UV{I} \sim \UV{G} \partial_z \UV{G}$ and $\UV{U} \sim \big[ \UV{\Sigma}, \UV{G} \big]$, we get
\begin{equation}
  \partial_z \UV{I} = \UV{U}.
\end{equation}
Writing out the Keldysh-space structure, the potential $\UV{U}$ becomes
\begin{align}
  \UV{U} &=
    \begin{pmatrix}
      \U\Sigma & 0 \\[0.5ex]
      0 & \U\Sigma
    \end{pmatrix}
    \begin{pmatrix}
      {\U G}^R & {\U G}^K \\[0.5ex]
      0        & {\U G}^A
    \end{pmatrix}
    -
    \begin{pmatrix}
      {\U G}^R & {\U G}^K \\[0.5ex]
      0        & {\U G}^A
    \end{pmatrix}
    \begin{pmatrix}
      \U\Sigma & 0 \\[0.5ex]
      0 & \U\Sigma
    \end{pmatrix}
  \\ &=
    \begin{pmatrix}
      \big[{\U\Sigma}\big, {\U G}^R] & \big[{\U\Sigma}, {\U G}^K\big] \\[1ex]
      0        & \big[{\U\Sigma}, {\U G}^A\big]
    \end{pmatrix}
\end{align}
In other words, as long as the self-energy is diagonal in Keldysh space, the Keldysh component of the matrix potential is simply
\begin{equation}
  {\U U}^K = \big[ {\U\Sigma}, {\U G}^K \big] .
\end{equation}
Now, the Pauli-decompositions of the Kelydysh components were:
\begin{align}
  \C{J}_i & \equiv \frac{1}{4} \tr\left\{ {\U \rho_i} {\U I}^K \right\}, &
  \C{U}_i & \equiv \frac{1}{4} \tr\left\{ {\U \rho_i} {\U U}^K \right\}.
\end{align}
Using the above equation for ${\U U}^K$ and the cyclic rule for traces:
\begin{align}
  \C{U}_i 
  &= \frac{1}{4} \tr\left\{ {\U \rho_i} \big[ {\U\Sigma}, {\U G}^K\big] \right\} \\
  &= \frac{1}{4} \tr\left\{ {\U \rho_i} \big[ {\U\Sigma} {\U G}^K - {\U G}^K {\U\Sigma} \big] \right\} \\
  &= \frac{1}{4} \tr\left\{ {\U \rho_i} {\U\Sigma} {\U G}^K - {\U\Sigma} {\U \rho_i} {\U G}^K \right\} \\
  &= \frac{1}{4} \tr\left\{ \big[{\U \rho_i}, {\U\Sigma}\big] {\U G}^K \right\}.
\end{align}
Inserting ${\U G}^K = {\U G}^R {\U H} - {\U H} {\U G}^A$:
\begin{align}
  \C{U}_i 
  &= \frac{1}{4} \tr\left\{ \big[{\U \rho_i}, {\U\Sigma}\big] {\U G}^R {\U H} - \big[{\U \rho_i}, {\U\Sigma}\big] {\U H} {\U G}^A \right\} \\
  &= \frac{1}{4} \tr\left\{ \big[{\U \rho_i}, {\U\Sigma}\big] {\U G}^R {\U H} - {\U G}^A \big[{\U \rho_i}, {\U\Sigma}\big] {\U H} \right\}.
\end{align}
As we did for the Pauli-decomposition of the current and potential, we can insert $\U H = \sum_j \C{H}_j {\U \rho}_j$, and move the scalar coefficients $\C{H}_j$ and summation out of the trace.
What we find then, is that the equation above can be written
\begin{align}
  \C{U}_i = \sum_j \C{K}_{ij} \C{H}_j,
\end{align}
where we define the new matrix coefficient
\begin{align}
  \C{K}_{ij}
  &= \frac{1}{4} \tr\left\{ \big[{\U \rho_i}, {\U\Sigma}\big] {\U G}^R {\U \rho}_j - {\U G}^A \big[{\U \rho_i}, {\U\Sigma}\big] {\U \rho}_j \right\}.
\end{align}
There are two things to note about this result.
First, this equation makes it almost trivial to specify the Usadel equation for new materials, since it is a direct function of the self-energy $\U \Sigma$ that we would put into the Usadel equation itself.
Second, the matrix $\U \Sigma$ is just filled with system parameters that we set (energy, exchange field, etc.), while the matrices ${\U G}^R$ and ${\U G}^A$ only depend on the equilibrium solution.
Since this matrix is completely independent of the nonequilibrium distribution $\BC H$, we can precalculate it before solving the differential equation, and do not have to evaluate it more than once per energy and position.



\section{Second-order terms}
Above, we considered a matrix potential $\V{U} = \big[\V{\Sigma}, \V{G}\big]$ where $\V{\Sigma}$ assumed to be independent of $\V{G}$.
This is sufficient for a lot of systems (e.g. normal metals, ferromagnets, even half-metals), but not completely general.
There are at least three exceptions I know of based on the projects I've worked on so far:\\[1ex]
\begin{tabular}{ll}
  Spin-flip scattering: &
  $\V{U} \sim \big[ \tau_3 \BU{\sigma} \V{G} \BU{\sigma} \tau_3, \V{G} \big];$ \\[0.5ex]
  Spin-orbit scattering: &
  $\V{U} \sim \big[ \phantom{\tau_3}\BU{\sigma} \V{G} \BU{\sigma} \phantom{\tau_3}, \V{G} \big];$ \\[0.5ex]
  Orbital depairing: & 
  $\V{U} \sim \big[  \tau_3 \phantom{\BU{\sigma}}\V{G} \phantom{\BU{\sigma}} \tau_3 , \V{G} \big]$.\\[1ex]
\end{tabular}

\noindent
Note that all of the above are second-order in the propagator~$\V{G}$, and has the same basic structure $\V{U} = \big[ \V{\Sigma} \V{G} \V{\Sigma}, \V{G} \big]$.
The matrix $\V{\Sigma}$ in this expression is again diagonal in Keldysh space and independent of the propagator $\V{G}$.
Let us now check what happens if we assume such a form for the potential~$\V{U}$:
\begin{align*}
  \UV{U}
  &=
    \big[ \V{\Sigma} \V{G} \V{\Sigma}, \V{G} \big] 
  \\ &=
    \begin{pmatrix}
      {\U\Sigma} {\U G}^R {\U\Sigma} &
      {\U\Sigma} {\U G}^K {\U\Sigma} \\
      0                              &
      {\U\Sigma} {\U G}^A {\U\Sigma} 
    \end{pmatrix}
    \begin{pmatrix}
      {\U G}^R & {\U G}^K \\[0.5ex]
      0        & {\U G}^A
    \end{pmatrix}
    -
    \begin{pmatrix}
      {\U G}^R & {\U G}^K \\[0.5ex]
      0        & {\U G}^A
    \end{pmatrix}
    \begin{pmatrix}
      {\U\Sigma} {\U G}^R {\U\Sigma} &
      {\U\Sigma} {\U G}^K {\U\Sigma} \\
      0                              &
      {\U\Sigma} {\U G}^A {\U\Sigma} 
    \end{pmatrix}
  \\ &=
    \begin{pmatrix}
      \big[{\U\Sigma} {\U G}^R {\U\Sigma}\big, {\U G}^R] & 
      {\U\Sigma} {\U G}^R {\U\Sigma} {\U G}^K            +
      {\U\Sigma} {\U G}^K {\U\Sigma} {\U G}^A            - 
      {\U G}^R {\U\Sigma} {\U G}^K {\U\Sigma}            -
      {\U G}^K {\U\Sigma} {\U G}^A {\U\Sigma}            \\[1ex]
      0                                                  &
      \big[{\U\Sigma} {\U G}^A {\U\Sigma}, {\U G}^A\big]
    \end{pmatrix}
\end{align*} 
In other words, the Keldysh component takes the more complicated form:
\begin{equation}
  \U{U}^K = 
      {\U\Sigma} {\U G}^R {\U\Sigma} {\U G}^K +
      {\U\Sigma} {\U G}^K {\U\Sigma} {\U G}^A - 
      {\U G}^R {\U\Sigma} {\U G}^K {\U\Sigma} -
      {\U G}^K {\U\Sigma} {\U G}^A {\U\Sigma} .
\end{equation}
Like before, we can multiply by ${\U\rho}_i$ and trace to obtain the Pauli-decomposition, then use the cyclic trace rule and reorder the terms a bit:
\begin{align*}
  \C{U}_i
  &= 
    \frac{1}{4}
    \tr 
    \left\{
      {\U\rho}_i {\U\Sigma} {\U G}^R {\U\Sigma} {\U G}^K +
      {\U\rho}_i {\U\Sigma} {\U G}^K {\U\Sigma} {\U G}^A - 
      {\U\rho}_i {\U G}^R {\U\Sigma} {\U G}^K {\U\Sigma} -
      {\U\rho}_i {\U G}^K {\U\Sigma} {\U G}^A {\U\Sigma} 
    \right\}
  \\ 
  &= 
    \frac{1}{4}
    \tr 
    \left\{
      {\U\rho}_i {\U\Sigma} {\U G}^R {\U\Sigma} {\U G}^K +
      {\U\Sigma} {\U G}^A {\U\rho}_i {\U\Sigma} {\U G}^K - 
      {\U\Sigma} {\U\rho}_i {\U G}^R {\U\Sigma} {\U G}^K -
      {\U\Sigma} {\U G}^A {\U\Sigma} {\U\rho}_i {\U G}^K 
    \right\}
  \\ 
  &= 
    \frac{1}{4}
    \tr 
    \left\{
      \big[ {\U\rho}_i, {\U\Sigma} \big] {\U G}^R {\U\Sigma} {\U G}^K +
      {\U\Sigma} {\U G}^A \big[ {\U\rho}_i, {\U\Sigma} \big] {\U G}^K 
    \right\}
\end{align*}
For simplicity, let's write this in the following form for now:
\begin{align}
  \C{U}_i
  &= 
    \frac{1}{4} \tr \left\{ \U F_i {\U G}^K \right\}
  ,
  &
  \U F_i
  = \big[ {\U\rho}_i, {\U\Sigma} \big] {\U G}^R {\U\Sigma} 
  + {\U\Sigma} {\U G}^A \big[ {\U\rho}_i, {\U\Sigma} \big]
  .
\end{align}
We then substitute in ${\U G}^K = {\U G}^R {\U H} - {\U H} {\U G}^A$:
\begin{align}
  \C{U}_i
  = 
    \frac{1}{4}
    \tr 
    \left\{
      {\U F}_i {\U G}^R {\U H} -
      {\U F}_i {\U H} {\U G}^A 
    \right\}
  = 
    \frac{1}{4}
    \tr 
    \left\{
      {\U F}_i {\U G}^R {\U H} -
      {\U G}^A {\U F}_i {\U H}
    \right\}
\end{align}
We then use $\U H = \sum_j H_j {\U\rho}_j$ to write the above equation as:
\begin{align}
  \C{U}_i    &= \sum_j \C{K}_{ij} \C{H}_j, &
  \C{K}_{ij} &= \frac{1}{4} \tr \left\{ {\U F}_i {\U G}^R {\U\rho}_j - {\U G}^A {\U F}_i {\U \rho}_j \right\}.
\end{align}
We could substitute back the definition of ${\U F}_i$ to see the explicit form of $\C{K}_{ij}$:
\begin{align*}
  \C{K}_{ij} = 
  \frac{1}{4} \tr 
  \Big\{
  &   \big[ {\U\rho}_i, {\U\Sigma} \big] {\U G}^R {\U\Sigma} {\U G}^R {\U\rho}_j
    + {\U\Sigma} {\U G}^A \big[ {\U\rho}_i, {\U\Sigma} \big] {\U G}^R {\U\rho}_j \\
  & - {\U G}^A \big[ {\U\rho}_i, {\U\Sigma} \big] {\U G}^R {\U\Sigma} {\U \rho}_j
    - {\U G}^A {\U\Sigma} {\U G}^A \big[ {\U\rho}_i, {\U\Sigma} \big] {\U \rho}_j
  \Big\}
\end{align*}
This expression has some symmetries.
Let us use the cyclic trace rule to rewrite it such that the commutator $\big[ {\U\rho}_i, {\U\Sigma} \big]$ comes first in each term:
\begin{align*}
  \C{K}_{ij} = 
  \frac{1}{4} \tr 
  \Big\{
  &   \big[ {\U\rho}_i, {\U\Sigma} \big] {\U G}^R {\U\Sigma} {\U G}^R {\U\rho}_j
    + \big[ {\U\rho}_i, {\U\Sigma} \big] {\U G}^R {\U\rho}_j {\U\Sigma} {\U G}^A \\
  & - \big[ {\U\rho}_i, {\U\Sigma} \big] {\U G}^R {\U\Sigma} {\U \rho}_j {\U G}^A
    - \big[ {\U\rho}_i, {\U\Sigma} \big] {\U \rho}_j {\U G}^A {\U\Sigma} {\U G}^A
  \Big\}
\end{align*}
It then becomes clear that this commutator can be factored out of all terms.
Furthermore, we see that the terms proportional to ${\U G}^R \cdots {\U G}^A$ have a commutator structure.
Thus, the above may be simplified to:
\begin{align*}
  \C{K}_{ij} = 
  \frac{1}{4} \tr 
  \Big\{
      \big[ {\U\rho}_i, {\U\Sigma} \big]
      \left( 
        {\U G}^R {\U\Sigma} {\U G}^R {\U\rho}_j
      - {\U \rho}_j {\U G}^A {\U\Sigma} {\U G}^A
      + {\U G}^R \big[{\U \rho}_j, {\U\Sigma} \big] {\U G}^A
      \right)
  \Big\}
\end{align*}
This solves the problem of second-order terms $\UV{U} \sim \big[\UV{\Sigma}\UV{G}\UV{\Sigma}, \UV{G}\big]$.

\section{Summary}
If we have a very general Usadel equation of the form:
\begin{align}
  iD \partial_z(\UV{G} \partial_z \UV{G}) = \big[ \UV{\Sigma}^{(1)} + \UV{\Sigma}^{(2)}\UV{G}\UV{\Sigma}^{(2)}, \UV{G} \big],
\end{align} 
I have shown that the Pauli-decomposed Usadel equation may be written:
\begin{equation}
  \BC{M} \partial_z^2 \BC{H} = -\big[ (\partial_z\BC{M}) + \BC{Q} \big] \partial_z \BC{H} - \big[ (\partial_z \BC{Q}) + \BC{K} \big] \BC{H}
\end{equation}
Where the contributions to $\BC{K}$ from the $\UV{\Sigma}^{(1)}$ and $\UV{\Sigma}^{(2)}$ terms are given by:
\begin{align*}
  \C{K}_{ij}^{(1)}
  &= 
  \frac{1}{4} \tr\left\{ \big[{\U \rho_i}, {\U\Sigma}^{(1)}\big] \left({\U G}^R {\U \rho}_j - {\U \rho}_j {\U G}^A\right) \right\},
  \\
  \C{K}_{ij}^{(2)}
  &= 
  \frac{1}{4} \tr 
  \Big\{
      \big[ {\U\rho}_i, {\U\Sigma}^{(2)} \big]
      \left( 
        {\U G}^R {\U\Sigma}^{(2)} {\U G}^R {\U\rho}_j
      - {\U \rho}_j {\U G}^A {\U\Sigma}^{(2)} {\U G}^A
      + {\U G}^R \big[{\U \rho}_j, {\U\Sigma}^{(2)} \big] {\U G}^A
      \right)
  \Big\}.
\end{align*}
Note that since these matrices $\BC{Q}$, $\BC{M}$, and $\BC{K}$ are all completely independent of the nonequilibrium distribution function $\BC{H}$ and its derivatives, these can be precomputed from the equilibrium solution.
In other words, the final version of the kinetic equation can be written:
\begin{equation}
  \partial_z^2\BC{H} = \BC{C}_1 \partial_z \BC{H} + \BC{C}_2 \BC{H},
\end{equation}
where $\BC{C}_1$ and $\BC{C}_2$ are straight-forward to compute from the self-energies ${\UV\Sigma}^{(1)}$ and ${\UV\Sigma}^{(2)}$ and equilibrium propagators ${\UV G}^R$ and ${\UV G}^A$, and these coefficients only need to be computed once.
This yields an explicit linear differential equation for $\BC{H}$, which can be solved very efficiently numerically, and which should also be relatively simple to implement.
In its current form, the equation above will likely be even faster to solve than the Riccati-parametrized Usadel equation for the retarded propagator.

\clearpage
\section{Numerics}
We had reformulated the differential equation as:
\begin{equation}
  \partial_z^2\BC{H} = \BC{C}_1 \partial_z \BC{H} + \BC{C}_2 \BC{H}.
\end{equation}
This can be written as a first-order ordinary differential equation:
\begin{equation}
  \frac{d}{dz}
  \begin{pmatrix}
    \!\phantom{'}\BC{H}'\, \\
    \!\BC{H}\,
  \end{pmatrix}
  = 
  \begin{pmatrix}
    \BC{C}_1 & \BC{C}_2 \\
    1        & 0
  \end{pmatrix}
  \begin{pmatrix}
    \!\phantom{'}\BC{H}' \\
    \!\BC{H}
  \end{pmatrix}
\end{equation}
Note that this gives us an explicit form for the Jacobian, which can be provided directly to the boundary value problem solver.
