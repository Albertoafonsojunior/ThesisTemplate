\chapter{Introduction}\noindent
In this thesis: study superconductivity and spintronics, both in and out of equilibrium.
Macroscopically, a \emph{superconductor} can be defined by one simple equation: 
\begin{equation}
  \B{E} = \B{B} = 0.
\end{equation}
In words: superconductors expel electromagnetic fields.
Both of these properties follow from a superconductors ability to mobilise dissipationless charge currents.

The fact that a superconductor expels electric fields ($\B{E}=0$) can be explained by it acting as a perfect electrical conductor.
When an external electric field is applied to the superconductor, the material responds by mobilising a dissipationless charge current that works to cancel the applied field.
This dissipationless charge current is of course the origin of the name ``superconductor''.
The fact that a superconductor also expels magnetic fields ($\B{B}=0$) means that it acts as a perfect diamagnet.
When an external magnetic field is applied, the material generates a dissipationless screening current that again prevents the magnetic field from penetrating it.
This is known as the \emph{Meissner effect}.


Superconductors: perfect electrical conductors ($\B{E} = 0$), and perfect diamagnets ($\B{B} = 0$).

What are they used for?
10mK / 1ns thermometer based on supercond:
\url{https://arxiv.org/pdf/1704.04762.pdf}

\lipsum
